
\documentclass[12pt]{report}
\usepackage[backref=page]{hyperref}
\usepackage{listings}
\usepackage{multirow}
\usepackage{graphicx}
\usepackage{geometry}
\usepackage{adjustbox}
\usepackage[T1]{fontenc}
\usepackage{lmodern}
\usepackage{xcolor}
\usepackage{ragged2e}
\usepackage{booktabs}
\usepackage{tikz}
\usetikzlibrary{arrows.meta, positioning, arrows, shapes.geometric}
\usepackage{array}
\geometry{a4paper, margin=1in}
\graphicspath{{images/}}

\hypersetup{
    colorlinks=true,
    linkcolor=black,
    filecolor=magenta,
    urlcolor=cyan,
    citecolor=blue,
    pdftitle={Project Plan},
    pdfpagemode=FullScreen,
}

\tikzstyle{start} = [rectangle, rounded corners, minimum width=3cm, minimum height=1cm,text centered, draw=black, fill=green!45]
\tikzstyle{stop} = [rectangle, rounded corners, minimum width=3cm, minimum height=1cm,text centered, draw=black, fill=red!45]
\tikzstyle{process} = [rectangle, minimum width=3cm, minimum height=1cm, text centered, draw=black, fill=cyan!30]
\tikzstyle{decision} = [diamond, minimum width=3cm, minimum height=1cm, text centered, draw=black, fill=purple!20]
\tikzstyle{arrow} = [thick,->,>=stealth]


\lstset{
    basicstyle=\ttfamily\small,
    breaklines=true,
    keywordstyle=\color{blue},
    stringstyle=\color{red},
    showstringspaces=false,
    literate=
        *{0}{{{\color{blue}0}}}{1}
         {1}{{{\color{blue}1}}}{1}
         {2}{{{\color{blue}2}}}{1}
         {3}{{{\color{blue}3}}}{1}
         {4}{{{\color{blue}4}}}{1}
         {5}{{{\color{blue}5}}}{1}
         {6}{{{\color{blue}6}}}{1}
         {7}{{{\color{blue}7}}}{1}
         {8}{{{\color{blue}8}}}{1}
         {9}{{{\color{blue}9}}}{1}
}

\lstdefinelanguage{json}{
  basicstyle=\ttfamily,
  morestring=[b]",
  moredelim=**[is][\color{red}]{\%\%}{\%\%},
  morekeywords={true,false,null}
}


\author{Stefan-Nikola Stanev}
\title{CloudCord - Research Plan}

\date{2024}

\begin{document}

\maketitle

\tableofcontents


\chapter{Problem}

In today’s digital landscape, real-time communication platforms like Discord and Slack are essential for both enterprise and community collaboration. However, existing solutions often face scalability issues, security vulnerabilities, and limited cloud-native integration. Enterprises require a highly available, secure, and scalable communication system that supports text, voice, and video while ensuring compliance with security best practices (e.g., OWASP, GDPR) and seamless DevOps integration.

CloudCord addresses these challenges by leveraging microservices, cloud-native deployment, and real-time data processing to create a reliable, secure, and scalable communication platform tailored for enterprise use.


\chapter{Opportunity}

With the growing demand for secure, scalable, and cloud-native communication platforms, there is an opportunity to build a modern alternative that integrates real-time messaging, voice, and video while ensuring enterprise-level security, compliance, and DevOps automation.

CloudCord offers a flexible, self-hosted, and cloud-integrated solution that can scale dynamically, meet strict security standards (e.g., OWASP, GDPR), and provide high availability through microservices and Kubernetes. By focusing on cloud-native best practices, distributed architecture, and CI/CD automation, CloudCord positions itself as an enterprise-ready alternative to existing platforms, catering to both businesses and communities.


\chapter{Research questions}

\section{Main question}

How can a cloud-native, microservices-based Discord-like platform be designed and implemented to ensure scalability, security, and real-time performance for enterprise use?

\section{Sub-Questions}

\begin{enumerate}

\item \textbf{Scalability:} What architectural patterns and cloud-native technologies can be used to ensure high availability and scalability in a microservices-based platform?

\item \textbf{DevOps:} How can CI/CD, infrastructure as code, and monitoring ensure reliable deployment, scalability and maintainability for CloudCord?

\item \textbf{Cloud Native:} How can CloudCord use cloud-native technologies to improve scalability, reliability, and cost efficiency?

\item \textbf{Security:} How can CloudCord integrate security best practices, such as authentication, encryption, and OWASP Top 10 mitigation, to protect user data and minimize vulnerabilities?

\item \textbf{Distributed Data:} How can CloudCord securely handle and store data while ensuring GDPR compliance and protecting sensitive information?
  
\end{enumerate}


\chapter{Research Methods}



\begin{table}[h!]
	\centering
	\begin{tabular}{|p{5cm}|p{3cm}|p{3cm}|p{3cm}|}
		\hline
		\textbf{Research Sub-Question} & \textbf{Library Research} & \textbf{Field Research} & \textbf{Workshop Research} \\ \hline
		Scalability       & Review scalability principles & Test scalability in real environments & Design scalable architecture \\ \hline
		DevOps             & Study DevOps tools and practices & Observe DevOps in production & Design CI/CD pipelines and tools \\ \hline
		Cloud Native      & Review cloud-native principles & Test cloud technologies in real environments & Prototype cloud-native solutions \\ \hline
		Security          & Study security best practices & Test security in real systems & Design security measures and prototypes \\ \hline
    	Distributed Data  & Review data storage solutions & Test data storage and compliance & Prototype data storage solutions \\ \hline
	\end{tabular}
	\caption{Research Methods for Each Research Sub-Question}
\end{table}





\chapter{Deliverables}

\begin{enumerate}

	\item\textbf{\large{Enterprise software solution}}
    
  \begin{enumerate}

    \item[] Software solution integrating cloud concepts and best practices for enterprise software 
    
  \end{enumerate}

	\item\textbf{\large{Documentation}}

    \begin{enumerate}

		\item Research plan

		\item Portfolio

		\item Other documents in the future

	\end{enumerate}


\end{enumerate}




\chapter{Planning}

\section*{Sprint 1: Research and Initial Design (Weeks 1-3)}
\begin{itemize}
    \item Research plan
    \item Initial design documentation, including architecture diagram and selected tools/technologies.
    \item Non-functional and functional requirements for CloudCord.
\end{itemize}

\section*{Sprint 2: Local Kubernetes Setup, Microservices, and Portfolio (Weeks 4-6)}
\begin{itemize}
    \item Local Kubernetes setup and configuration.
    \item Initial microservices (user authentication, basic functionality).
    \item Portfolio documentation with progress so far.
\end{itemize}

\section*{Sprint 3: Security, Additional Microservices, CI/CD, and Portfolio (Weeks 7-9)}
\begin{itemize}
    \item Security implementation (authentication, encryption, OAuth).
    \item Additional microservices integrated into the system.
    \item Working CI/CD pipeline (with automated testing).
    \item Portfolio with added security features and CI/CD documentation.
\end{itemize}

\section*{Sprint 4: Distributed Data, Final Microservices, and Portfolio (Weeks 10-12)}
\begin{itemize}
    \item Distributed data management features (e.g., cloud storage, data replication).
    \item Final microservices integrated into the platform.
    \item Cloud-native integrations (e.g., Kubernetes).
    \item Portfolio with detailed implementation of distributed data and final microservices.
\end{itemize}

\section*{Sprint 5: Deployment, Scalability, and Portfolio (Weeks 13-15)}
\begin{itemize}
    \item Cloud deployment of CloudCord platform.
    \item Scalability and performance testing results.
    \item Updated portfolio with deployment and scalability insights.
\end{itemize}

\section*{Sprint 6: Final Solution, Presentation, and Final Portfolio (Weeks 16-18)}
\begin{itemize}
    \item Fully deployed and functional CloudCord platform.
    \item Final project report and presentation.
    \item Completed and polished portfolio with all documentation, research, design, and results.
\end{itemize}



\end{document}



